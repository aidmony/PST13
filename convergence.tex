

\section{Convergence Model} \label{sec:mathmodel}
Researchers have long argued that every social network has a tendency
towards a balanced state~\cite{Doreian:02}.  The next question of
interest is if an imbalance rises, in what way will a social network
change towards a new balance.  It is noted in social psychology
literature that people are reluctant to make changes in relations as
they tend to avoid the effort needed to make such changes. In a balanced
triadic relation, participants are likely to do nothing and keep their
pairwise relations as what they were. In an imbalanced triadic
relation, participants are likely to make the smallest effort possible
to regain triadic balance. We define the concept of relation cost as
the effort one needs to take to accomplish a certain relation
change. Our convergence model is established based on a unified
assumption: every social network converges in a way that requires as
little total change in relations as possible to reach a balanced
state.

With the concept of relation distance, we are able to express the
structure of a social network by drawing it in the Euclidean
space. The strength of each relation is expressed by the distance
between their locations. Notice that every layout in the Euclidean
space automatically satisfies the metric triangle inequality, and
hence corresponds to a balanced state of the network.  For an
imbalanced social network, it is not possible to draw it using its
initial relation distances.  Hence, our convergence model produces a
layout of the social network with minimum total relation cost from the
original one.

Let $G=(V,E)$ denote an arbitrary social network, and $G^{*}=(V,
E^{*})$ denote a balanced state of $G$. Let $n*n$ matrix $X$ denote
the layout of $G^{*}$, with each row vector $x_{i}$ denoting node
$i$'s location in $m$-dimensional space. For each pair $(i,j)$,
$\psi(i,j)$ denotes its relation distance in $G$, and $d_{i,j}(X)$
denotes distance between $i$ and $j$ in $X$, i.e., its relation
distance in $G^{*}$.  Given an edge $(i,j)\in E$, the
relation cost on $(i,j)$ is given by:
\[c_{i, j}(X)=w_{\psi(i,j)}*(d_{i,j}(X)-\psi{(i,j)})^2\]
where the weight value is a function of the original distance. The
weight function can take into account the difficulty of changing a
relation. For example, it is generally easier to change a neutral
relation than a positive or a negative relation that incorporate an
initial bias. The study of optimal weights is beyond the scope of this
paper. However, we consider three main classes of weights:
\[
 w_{\psi(i,j)}= \left\{ 
  \begin{array}{l l}
    w_{+} & \quad \text{if $\psi(i,j)$ is a positive edge}\\
    w_{O} & \quad \text{if $\psi(i,j)$ is a neutral edge}\\
    w_{-} & \quad \text{if $\psi(i,j)$ is a negative edge}\\
  \end{array} \right.
\]

If $w_{O} << w_{+}$ and $w_{O} << w_{-}$, then neutral edges would
have very little influence on the already established
positive/negative relations.

\begin{definition}
Let $G=(V,E)$ be a social network where $E$ is a set of weighted
edges. Its converged network $G^{*}=(V,E^{*})$ is given by layout
matrix $X$ with $d_{i,j}(X)$ as the relation distance between every
pair $(i,j)$, such that the total relation cost $\sigma(X)$ is
minimized:
\[\sigma(X)= \min_{X} \sum_{i<j \leq n}w_{\psi(i,j)}*(d_{i,j}(X)-\psi{(i,j)})^2\]
\end{definition}

The optimization of relation cost is in fact a Metric Multidimensional
Scaling problem (MDS) by assigning nodes a location in metric
space. The total cost function is called stress in MDS, and is often
minimized through an optimization strategy called {\it stress
  majorization} \cite {Gansner:05}. Stress majorization is an
iterative method that guarantees monotonically decreasing stress in
each iteration, and returns a locally minimum solution.  It is
recognized as a principled technique in the field of graph
drawing. The algorithm, however, requires $O(n^3)$ time and $O(n^{2})$
space. Due to its complexity, stress majorization is applicable on
graphs with limited size when missing edges are explicitly represented
as neutral ties. We are in the process of developing an approximation
method for this problem as a result. In this paper, however, we will
investigate the performance of the exact solution to stress
majorization.


